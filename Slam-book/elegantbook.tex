\documentclass{elegantbook}

\author{Miao}
\date{\today}
\email{chenmiao.ku@gmail.com}
\usepackage{ntheorem}
\zhtitle{基于视觉十四讲}
\entitle{Slam Note}
\enend{笔记}
\version{0.10}
\myquote{Victory won\rq t come to us unless we go to it.}
\logo{ElegantLaTeX_green.pdf}
\cover{cover.pdf}

\usepackage{listings}
\usepackage{xcolor}
\usepackage{makecell}
\usepackage{lipsum}
\usepackage{texnames}

\lstset{ 
  backgroundcolor=\color{white},   % 选择代码背景,必须加上\ usepackage {color}或\ usepackage {xcolor}.
  basicstyle=\bf,                  % 设置代码字号.
  breakatwhitespace=false,         % 设置是否当且仅当在空白处自动中断.
  breaklines=true,                 % 设置自动断行.
  captionpos=b,                    % 设置标题位置.
  commentstyle=\color{red},    % 设置注释格式
  deletekeywords={...},            % 是否删除给定语言的关键词.
  escapeinside={\%*}{*)},          % 是否在代码中添加LaTex.
  extendedchars=true,              % 是否允许使用非ASCII字符; 仅适用于8位编码,不适用于UTF-8. 
  frame=single,	                   % 给代码区添加边框.
  keepspaces=true,                 % 保留空格(useful for keeping indentation of code (possibly needs columns=flexible).
  keywordstyle=\color{blue},       % 关键字显示风格.
  language=C++,                    % 使用的语言.
  morekeywords={*,...},            % 是否需要添加其他的关键词.
  numbers=left,                    % 给代码添加行号,可取值none, left, right.
  numbersep=5pt,                   % 设置行号与代码之间的间隔
  numberstyle=\bf\color{blue},     % 行号的字号和颜色
  rulecolor=\color{black},         % 边框颜色,如果没有设置,框架颜色可以在非黑色文本中的换行符上更改(例如 text (e.g. comments (green here)))
  showspaces=false,                % 显示每个地方添加特定下划线的空格; 覆盖了'showtringspaces'
  showstringspaces=false,          % 仅在字符串中允许空格
  showtabs=false,                  % show tabs within strings adding particular underscores
  stepnumber=2,                    % the step between two line-numbers. If it's 1, each line will be numbered
  stringstyle=\color{green},     % string literal style
  tabsize=4,	                   % 将默认tab设置为2个空格
  title=\lstname                   % show the filename of files included with \lstinputlisting; also try caption instead of title
}


\begin{document}
    \maketitle
    \tableofcontents
    \chapter{计算机系统概述}

\section{操作系统的基本概念}

\subsection{操作系统的概念}

    \emph{操作系统(Operator System)是指控制和管理整个计算机系统的{\color{red}硬件和软件资源},合理地组织、调度计算机的工作与资源的分配,进而为用户和其他软件提供方便接口与环境的程序集合。}

    \emph{\color{red}操作系统是计算机系统中最基本的系统软件。}

\subsection{操作系统的特征}

    操作系统有如下四大特征:

\begin{itemize}
    \item [1.] 并发(Concurrence)
    \subitem 并发是指两个或多个事件在同一时间间隔内发生。值得注意的是:\emph{并发和并行是两个不同的概念,{\color{red}并行一定并发,并发不一定并行。对于单处理器来说,只能并发执行。}}
    \subitem \emph{并发是OS最为基础且必要的特征。}
    \item [2.] 共享(Sharing)
    \subitem 共享即资源共享,是建立在并发之上的。
    \subitem (互斥式共享) 系统中的某些资源在一个时间段内有且仅能一个程序进行访问和操作,且把\emph{{\color{red}该资源称为临界资源。}}
    \subitem (同时式共享) 该类资源可以由多个程序同时段进行访问(多出现在读操作上)
    \subitem \emph{并发和共享是操作系统{\color{red}最基本}的特征,两者之间互为存在的条件:资源共享必定是由并发产生的,而共享影响了并发则会导致并发崩溃。}
    \item [3.] 虚拟(Virtual)
    \subitem \emph{虚拟是指把一个物理上的实体变为若干逻辑上的对应物。{\color{red}物理实体是实际存在的,而虚拟逻辑是用户感觉上的事物。}}
    \subitem 虚拟处理器的存在:时分复用技术
    \subitem 虚拟内存的存在:空分复用技术
    \item [4.] 异步(Asynchronism)
    \subitem 多道程序环境允许多个程序执行,但由于资源有限(竞争的情况),进程并非一贯到底的执行。而且走走停停,不停的切换,因此以不可预知的速度向前推进。
\end{itemize}

\subsection{操作系统的目标与功能}

    \emph{为了给多道程序提供环境,OS应该具备:处理机(进程)管理、存储器(内存)管理、设备管理和文件管理(现代操作系统中,不仅仅这几个模块)。}

    \emph{除了上述的模块外,还需要提供各种接口。}

\subsubsection{操作系统作为计算机系统资源的管理者}

\begin{itemize}
    \item [1)] 处理机(进程)管理
    \subitem 在多道程序环境下,处理机的分配和运行都以进程(或线程)为基本单位,因此对处理机管理可归结为进程管理。
    \subitem 并发指的是\emph{计算机内同时运行多个进程,因此如何管理进程则是最主要的任务。进程管理的主要功能包括进程控制、进程同步、进程通信、死锁处理、处理机调度等。}
    \item [2)] 存储器(内存)管理
    \subitem 内存管理是为了给多道程序的运行提供良好的环境,方便用户使用以及提高利用率。\emph{其主要包括内存分配与回收、地址映射、内存保护与共享和内存扩充等。}
    \item [3)] 文件管理
    \subitem \emph{{\color{red}计算机中的信息都是以文件的形式存在的},文件管理包括文件存储空间的管理、目录管理以及文件读写管理与保护等。}
    \item [4)] 设备管理
    \subitem 设备管理的主要任务是完成用户的I/O请求,方便用户使用各种设备,提高设备利用率。主要包括缓冲管理、设备分配、设备处理和虚拟设备等。
\end{itemize}

\subsubsection{操作系统作为用户与计算机硬件系统之间的接口}

\begin{itemize}
    \item [1)] 命令接口
    \subitem (联机命令接口(交互式)) \emph{{\color{red}适用于分时或实时操作系统。}由一组键盘命令组成,用户通过控制台或terminal输入命令与OS进行交互。}类似于Python交互器。
    \subitem (脱机命令接口(批处理)) \emph{{\color{red}适用于批处理系统。}由一组作业控制命令组成,不能直接干预作业的运行,事先使用对应的作业控制命令写成一份操作流程,然后递交给OS。}类似于Bash脚本。
    \item [2)] 程序接口
    \subitem 程序接口由系统调用(又称广义指令)组成,例如printf、malloc等C语言调用接口。
    \subitem GUI就是图形接口,其最终是通过调用程序接口实现的。{\color{red}严格的说,GUI不是操作系统的一部分,但GUI所使用的系统调用是OS的一部分。}
\end{itemize}

\subsubsection{操作系统实现了对计算机资源的扩展}

    \emph{没有{\color{red}任何软件支持的}计算机被称为裸机,其仅构成计算机系统的物质基础。}

    也就是说,操作系统的内部是各种物理结构所组成的逻辑环境,操作系统所提供的资源管理功能和方便用户的各种服务功能,将裸机改造成功能更强、更方便的机器。\emph{通常,把覆盖了软件的机器称为扩充机器或虚拟机。}

\section{操作系统发展历程}

\subsection{手工操作系统}

    \emph{此阶段并未产生严格意义上的OS。} 

    手工操作系统的突出缺点:\emph{用户独占全机,资源利用效率极低;CPU等待手工操作,CPU利用效率极低。}

\subsection{批处理阶段}

    \emph{为了解决人机矛盾以及I/O设备之间速度不匹配的问题,出现了批处理系统。}

\subsubsection{单道批处理系统}

    系统对作业的处理是成批进行的,但{\color{red}内存中始终只保存一道作业。}其主要特征为:

\begin{itemize}
    \item [1)] 自动性。磁带上的一批作业自动逐个进行,无需人工干预
    \item [2)] 顺序性。磁带上的作业按顺序进入内存
    \item [3)] 单道性。内存中仅有一道作业运行,即监督程序每次从磁带上只调入一道程序。
\end{itemize}

    单道批处理系统的主要问题在于:\emph{内存中仅有一道作业,CPU有大量的时间是在等待I/O的完成。}

\subsubsection{多道批处理系统}

    为了解决资源利用率和系统吞吐量,引入了多道程序技术。\emph{多道程序技术允许多个程序同时进入内存并允许在CPU中交替运行,共享系统中的各种软硬件资源。}

    其设计的特点为\emph{多道、{\color{red}宏观上并行,微观上串行。}}但是,多道程序设计需要解决:

\begin{itemize}
    \item [1)] 如何分配处理器
    \item [2)] 如何分配内存
    \item [3)] 如何分配I/O设备
    \item [4)] 如何组织和存放大量的程序和数据,且保证数据安全和一致性
\end{itemize}

    其优点在于:\emph{资源利用率高,且共享计算机资源从而使各种资源得到充分利用;系统吞吐量大,CPU和其他资源保持“忙碌”状态。}

    缺点在于:\emph{用户响应时间较长;不提供人机交互能力,用户既不能了解运行情况也不能控制计算机。}

\subsubsection{分时操作系统}

    \emph{分时技术,也就是将处理器的运行时间分成很短的时间片,按照时间片轮流把处理器分配给各联机作业使用。{\color{red}这使得每个用户(程序)感觉起来就像自己独占一台计算机}。}

    \emph{分时操作系统,指的是多个用户通过终端同时共享一台主机,用户可以同时与主机进行交互操作而互不干扰。}因此,分时系统最关键的问题在于:{\color{red}如何使用户与自己的作业交互}。分时系统支持多道程序设计,但不同于多道批处理,分时系统是人机交互的系统:

\begin{itemize}
    \item [1)] 同时性。\emph{同时性又称多路性,允许多个终端用户同时使用一台计算机。}
    \item [2)] 交互性。用户能够与系统进行人机对话。
    \item [3)] 独立性。系统中的各个用户独立操作,互不干扰。
    \item [4)] 及时性。用户请求能在很短时间内响应,采用时间片轮转算法。
\end{itemize}

\subsubsection{实时操作系统}

    \emph{为了满足某个时间限制内完成某些紧急任务而不需要时间片排队,诞生了实时操作系统。}

    硬实时系统:\emph{某个动作必须绝对地在规定的时刻内完成或发生。}

    软实时系统:\emph{能够接收偶尔违反时间规定且不会引起任何永久性的损害。}

    在实时操作系统的控制下,计算机系统接收到外部信号后及时处理,并在严格时间内处理完毕,其主要特点就是{\color{red}及时性和可靠性}。

    \chapter{三维空间刚体运动}

\section{旋转矩阵} 

\subsection{点和向量,坐标系}

    如何表示一个点在三维空间,假设一个线性空间的基为:$(e_1, e_2, e_3)$,那么:

$$
a = 
\begin{bmatrix}
e_1, & e_2, & e_3
\end{bmatrix}
\begin{bmatrix}
a_1 \\
a_2 \\
a_3 
\end{bmatrix} = a_1e_1 + a_2e_2 + a_3e_3
$$

    对于向量的内积:

$$
    a \cdot b = a^Tb = \sum_{i = 1}^3{a_ib_i} = |a||b|cos<a, b>
$$

    对于向量的外积:

$$
a \times b = 
\begin{bmatrix}
i & j & k \\
a_1 & a_2 & a_3 \\
b_1 & b_2 & b_3 
\end{bmatrix} = 
\begin{bmatrix}
a_2b_3 - a_3b_2 \\
a_3b_1 - a_1b_3 \\
a_1b_2 - a_2b_1
\end{bmatrix} = 
\begin{bmatrix}
0 & -a_3 & a_2 \\
a_3 & 0 & -a_1 \\
-a_2 & a_1 & 0
\end{bmatrix}b \triangleq a^\land b
$$

    外积的方向垂直于这两个方向,大小为$|a||b|sin<a, b>$。对于外积,此处引入了${}^\land$符号,把$a$携程一个矩阵。事实上是一个反对称矩阵($Skew-symmetric$),可以将${}^\land$记成一个反对称符号。

    外积只对三维向量存在定义,可以用外积表示向量的旋转

\subsection{坐标系间的欧式变换}

\begin{quote}
    \centering
    描述两个坐标系之间的旋转关系,加上平移统称为坐标系之间的变换关系
\end{quote}

\begin{figure}[!htbp]
    \centering
    \includegraphics[width=0.3\textwidth]{image/chapter02/坐标系的旋转.png}
    \caption{坐标系的旋转}
\end{figure}

    \emph{相机运动是一个刚体运动,保证了同一个向量再各个坐标系下的长度和角度不会发生变化},这被称为欧式变换。

    一个欧式变换由\emph{一个旋转和一个平移两部分组成}。首先考虑旋转,我们设某个单位正交基$(e_1, e_2, e_3)$经过一次旋转变为$(e_1^{'}, e_2^{'}, e_3^{'})$,那么对于同一个向量$a$(\emph{该向量并没有随着坐标系的旋转而发生运动}),它再这两个坐标系下的坐标分别为:$\begin{bmatrix}a_1, & a_2, & a_3\end{bmatrix}^T$和$\begin{bmatrix}a_1^{'}, & a_2^{'}, & a_3^{'}\end{bmatrix}^T$,那么就有:

$$
\begin{bmatrix}
    e_1, & e_2, & e_3
\end{bmatrix}
\begin{bmatrix}
    a_1 \\ a_2 \\ a_3
\end{bmatrix} = 
\begin{bmatrix}
    e_1^{'}, & e_2^{'}, & e_3^{'}
\end{bmatrix}
\begin{bmatrix}
    a_1^{'} \\ a_2^{'} \\ a_3^{'}
\end{bmatrix}
$$

    此时将上述等式的左右两边同时乘上$\begin{bmatrix}e_1^{T} \\ e_2^{T} \\ e_3^{T}\end{bmatrix}$:

$$
\begin{bmatrix}
    a_1 \\ a_2 \\ a_3
\end{bmatrix} = 
\begin{bmatrix}
    e_1^Te_1^{'}, & e_1^Te_2^{'}, & e_1^Te_3^{'} \\
    e_2^Te_1^{'}, & e_2^Te_2^{'}, & e_2^Te_3^{'} \\
    e_3^Te_1^{'}, & e_3^Te_2^{'}, & e_3^Te_3^{'} 
\end{bmatrix}
\begin{bmatrix}
    a_1^{'} \\ a_2^{'} \\ a_3^{'}
\end{bmatrix} \triangleq Ra^{'}
$$

    我们把中间的矩阵拿出来,定义成一个矩阵$R$。\emph{这个矩阵由两组基之间的内积组成,刻画了旋转前后同一个向量的坐标变换关系}。只要旋转是一样的,这个矩阵也是一样的。\emph{可以说,矩阵$R$描述了旋转本身,因此又称为旋转矩阵}

    旋转矩阵本身有一些特别的性质,比如它是一个行列式为1的正交矩阵,反之,行列式为1的正交矩阵也是一个旋转矩阵。因此,可以定义:

$$
	SO(n) = \{R \in \mathbb{R}^{n \times n} | RR^T = I, det(R) = 1 \}
$$

    $SO(n)$是特殊正交群($Special Orthogonal Group$)的意思(下一讲)。\emph{旋转矩阵可以描述相机的旋转},而$R$满足以下性质:


$$
\begin{aligned}
	a^{'} &= R^{-1}a = R^Ta \\
	a_1 &= R_{12}a_2 \\
	a_3 &= R_{32}a_2 = R_{32}R_{21}a_1 = R_{31}a_1
\end{aligned}
$$

    最后考虑世界坐标系中的向量$a$经过一次旋转和一次平移$t$后,得到$a^{'}$:

$$
    a^{'} = Ra + t
$$

    其中,$t$被称为平移向量,相比于旋转,平移部分只需要把这个平移量加到旋转后的坐标上。

\subsection{变换矩阵与齐次坐标}

    假定在上述给出的式子上我们进行了两次变换:$R_1,t_1$和$R_2,t_2$:

$$
    b = R_1a + t_1 ,\quad c = R_2b + t_2
$$

    那么就能够得到从$a$到$c$的变换:

$$
    c = R_2(R_1a + t_1) + t_2
$$

    这样的形式在多次变换后会很复杂,因此引入齐次坐标和变换矩阵重写:

$$
\begin{bmatrix}
    a^{'} \\ 1
\end{bmatrix} = 
\begin{bmatrix}
    R & t \\ 
    0^T & 1
\end{bmatrix}
\begin{bmatrix}
    a \\ 1
\end{bmatrix} \triangleq T 
\begin{bmatrix}
    a \\ 1
\end{bmatrix}
$$

    用这四个坐标表示三维向量的做法称为齐次坐标,引入齐次坐标后,旋转和平移可以放入同一个矩阵,称为变换矩阵,记作$T$矩阵。那么,经过多次变换后,通过变换矩阵可以得出:

$$
    \tilde{b} = T_1\tilde{a}, \tilde{c} = T_2\tilde{b} \Rightarrow \tilde{c} = T_2T_1\tilde{a}
$$

    对于变换矩阵,具有比较特别的结构:左上角为旋转看矩阵,右侧为平移向量,左下角为零向量,右下角为1。这种矩阵又称为特殊欧式群($Special Euclidean Group$)

$$
se(3) = \{ T = 
\begin{bmatrix}
    R & t \\
    0^T & 1
\end{bmatrix} \in \mathbb{R}^{4 \times 4} | R \in SO(3), t \in \mathbb{R}^3
\}
$$

    那么可以因此求得该矩阵的一个反向的变换:

$$
T^{-1} = 
\begin{bmatrix}
    R^T & -R^Tt \\
    0^T & 1
\end{bmatrix}
$$

\subsubsection{例子}

\begin{quote}
    \centering
    在Slam中,通常定义世界坐标系$T_W$与机器人坐标系$T_R$
\end{quote}

    一个点的世界坐标系为$p_W$,机器人坐标系下为$p_R$,则有:$p_R = T_{RW}p_W$。 也就是说,机器坐标系的点$p_R$可以由世界坐标系下的点$p_W$通过$T_{RW}$变换得到

\section{Eigen}

    对于Eigen的安装来说,这是一件非常容易的事(指在Linux操作系统或类Unix操作系统上),我们只需要键入以下命令:

\begin{lstlisting}[language=C++]
    sudo apt-get install libeigen3.dev
\end{lstlisting}

    对于Eigen第三方库的使用来说,也是较为简便的,因为Eigen只有头文件,是的,因此我们可以在\emph{CMakeLists.txt}文件中使用以下条件命令来引入:

\begin{lstlisting}[language=C++]
    find_package(Eigen3 REQUIRED)
    include_directories(${EIGEN3_INCLUDE_DIRS})
\end{lstlisting}

\subsection{Eigen的使用示例}

    对于Eigen来说,其使用是需要了解Eigen库的,此处我们使用一个小的例子来说明如何使用Eigen

\begin{lstlisting}[language=C++]
#include <iostream>
#include <ctime>
#include "eigen3/Eigen/Core"
#include "eigen3/Eigen/Dense"

#define MATRIX_SIZE 50

int main() {
    Eigen::Matrix<float, 2, 3> matrix_23;
    Eigen::Vector3d v_3d;
    Eigen::Matrix3d matrix_33 = Eigen::Matrix3d::Zero();
    Eigen::Matrix<double, Eigen::Dynamic, Eigen::Dynamic> matrix_dynamic;
    Eigen::MatrixXd matrix_x;

    matrix_23 << 1, 2, 3, 4, 5, 6;
    std::cout << matrix_23 << std::endl;

    for (int i = 0; i < 1; i++) {
        for (int j = 0; j < 2; j++)
            std::cout << matrix_23(i, j) << " ";
        std::cout << "\n";
    }

    v_3d << 3, 2, 1;
    // Eigen::Matrix<double, 2, 1> result_wrong_type = matrix_23 * v_3d;
    Eigen::Matrix<double, 2, 1> result = matrix_23.cast<double>() * v_3d;
    std::cout << result << std::endl;

    matrix_33 = Eigen::Matrix3d::Random();
    std::cout << matrix_33 << std::endl;
    std::cout << matrix_33.transpose() << std::endl;
    std::cout << matrix_33.sum() << std::endl;
    std::cout << matrix_33.trace() << std::endl;
    std::cout << 10 * matrix_33 << std::endl;
    std::cout << matrix_33.inverse() << std::endl;
    std::cout << matrix_33.determinant() << std::endl;

    Eigen::SelfAdjointEigenSolver<Eigen::Matrix3d> eigen_solver(matrix_33.transpose() * matrix_33);

    std::cout << "Eigen value: " << eigen_solver.eigenvalues() << std::endl;
    std::cout << "Eigen vectors: " << eigen_solver.eigenvectors() << std::endl;

    Eigen::Matrix<double, MATRIX_SIZE, MATRIX_SIZE> matrix_NN;
    matrix_NN = Eigen::MatrixXd::Random(MATRIX_SIZE, MATRIX_SIZE);
    Eigen::Matrix<double, MATRIX_SIZE, 1> v_Nd;
    v_Nd = Eigen::MatrixXd::Random(MATRIX_SIZE, 1);

    clock_t tiem_stt = clock();
    Eigen::Matrix<double, MATRIX_SIZE, 1> x = matrix_NN.inverse() * v_Nd;
    std::cout << "tiome use in normal invers is: " << 1000 * (clock() - tiem_stt) / (double)CLOCKS_PER_SEC << "ms" << std::endl;

    tiem_stt = clock();
    x = matrix_NN.colPivHouseholderQr().solve(v_Nd);
    std::cout << "tiome use in Qr compsition invers is: ";

    return 0;
\end{lstlisting}

\section{旋转向量和欧拉角}

\subsection{旋转向量}

    我们回到理论部分,探究矩阵表示方式中的缺点:

\begin{itemize}
    \item [1)] $SO(3)$的旋转矩阵由9个量,但一次旋转只有三个自由度,因此是冗余的
    \item [2)] 旋转矩阵自身带有约束:必须是正交矩阵,且行列式为1
\end{itemize}

    我们希望有一种方式能够紧凑的描述旋转和平移,我们知道\emph{任意旋转都可以用一个旋转轴和一个旋转角}来刻画,语句我们使用一个向量,\emph{其方向与旋转轴一致,而长度等于旋转角,这种向量被称为旋转向量(或轴角,$Axis-Angle$)}

$$
    w = \theta{n}
$$

    这是下一节中的李代数,因此目前只需要了解这样表示即可。之后,从旋转向量到旋转矩阵的转换过程由罗德里格斯公式($Rodrigues's Formula$)表明:

$$
    R = cos\theta{I} + (1 - cos\theta)nn^T + sin\theta{n^\land}
$$

    符号$\land$是向量到反对称的转换符,反之,我们也可以有一个旋转矩阵到旋转向量的转换:

$$
\begin{aligned}
    tr(R) &= cos\theta{tyr(I)} + (1 - cos\theta)tr(nn^\land) + sin\theta{tr(n^\land)} \\
        &= 3cos\theta + (1 - cos\theta) \\
        &= 1 + 2cos\theta
\end{aligned} \\
$$

    对于转轴$n$,由于旋转轴上的向量在旋转后不发生改变,说明:

$$
    Rn = n
$$

\subsection{欧拉角}

\begin{quote}
    \centering
    无论是旋转矩阵、旋转向量对于人来说不是很直观,因此,欧拉角使用了\emph{三个分离的转角}
\end{quote}

\begin{figure}[!htbp]
    \centering
    \includegraphics[width=0.6\textwidth]{image/chapter02/欧拉角.png}
    \caption{欧拉角}
\end{figure}

    在欧拉角中,常用的一种是“偏航--俯仰--滚转”($yaw--pitch--roll$)三个角度来描述一个旋转。

\begin{itemize}
    \item [1)] 绕物体的$Z$轴旋转,得到偏航角$yaw$
    \item [2)] 绕物体的$Y$轴旋转,得到俯仰角$pitch$
    \item [3)] 绕物体的$X$轴旋转,得到滚转角$roll$
\end{itemize}
    
\begin{figure}[!htbp]
    \centering
    \includegraphics[width=0.4\textwidth]{image/chapter02/ZYX欧拉角.png}
    \caption{Z-Y-X欧拉角}
\end{figure}

    此时,可以使用$\begin{bmatrix}r, & p, & y \end{bmatrix}^T$这样的一个三维向量表示任意旋转。

    但是,\emph{欧拉角的一个重大缺点是会碰见著名的万向锁问题($Gimbal Lock$):在俯仰角为$\pm90^{\circ}$时,第一次旋转与第三次旋转将使用同一个轴,使得系统丢失了一个自由度,这被称为奇异性问题}
    
    由于这种原理,欧拉角不适于插值和迭代,往往只适用于人机交互

\begin{figure}[!htbp]
    \centering
    \includegraphics[width=0.6\textwidth]{image/chapter02/万向锁.png}
    \caption{万向锁问题}
\end{figure}

\section{四元数}

\subsection{四元数定义}

    旋转矩阵用9个量描述三个自由度的旋转,但具有冗余性;欧拉角和旋转向量是紧凑的,但具有奇异性。因此,我们需要提出一种新的方式来描述旋转

    回想一下复数:复数的惩罚表示复平面上的旋转。\emph{因此,表达三维空间旋转时,有一种类似复数的代数:四元数($Quaternion$),四元数既是紧凑的,也没有奇异性,唯独不够直观,计算稍微复杂}

    一个四元数$q$拥有一个实部和三个虚部:$q = q_0 + q_1i + q_2j + q_3k$。其中,$i, j, k$为四元数的三个虚部,三个虚部满足:

$$
\begin{cases}
    i^2 = j^2 = k^2 = -1 \\
    ij = k, ji = -k \\
    jk = i, kj = -i \\
    ik = j, ki = -j 
\end{cases}
$$

    我们可以简单的将虚部看作三个转轴向量,但是实际上不是。由于这种特殊的表示形式,我们可以用一个标量和一个向量来表示四元数:

$$
    q = [s, v], s = q_0 \in \mathbb{R}, v = [q_1, q_2, q_3]^T \in \mathbb{R}^3
$$

    这里,$s$称为四元数的实部,$v$称为它的虚部。

\begin{figure}[!htbp]
    \centering
    \includegraphics[width=0.5\textwidth]{image/chapter02/二维向量乘法.png}
    \caption{二维向量的乘法}
\end{figure}

    可以看见,从1到-1的过程实际上是旋转了$180^{\circ}$,如果乘以虚数$i$,就相当于绕原点逆时针$90^{\circ}$

\begin{figure}[!htbp]
    \centering
    \includegraphics[width=0.6\textwidth]{image/chapter02/二维向量加减.png}
    \caption{二维向量的加法}
\end{figure}

    对于加减,相当于对轴上进行扩张和缩减,对于乘除,就相当于原点不变,进行缩放和扩张以及旋转

\begin{figure}[!htbp]
    \centering
    \includegraphics[width=0.6\textwidth]{image/chapter02/四元数到三维的变换模型.png}
    \caption{四元数从四维到三维的变换}
\end{figure}

    因此,对于四元数来说,可以通过四维来表示三维的旋转,$ij = k, ji = -k, i^2 = -1$是必须存在的条件,同时也会随着旋转被映射到无穷远最终回到三维

    同时,我们发现,旋转两个周期才会回到与原先的样子相等。假设某个旋转是绕单位向量$n = [n_x, n_y, n_z]^T$进行了角度为$\theta$的旋转,那么这个旋转可以表示为:

$$
    q = [cos\frac{\theta}{2}, n_xsin\frac{\theta}{2}, n_ysin\frac{\theta}{2}, n_zsin\frac{\theta}{2}]^T
$$

    反之,亦可以得到对应旋转轴与夹角:

$$
\begin{cases}
    \theta = 2arccosq_0 \\
    [n_x, n_y, n_z]^T = [q_1, q_2, q_3]^T / sin\frac{\theta}{2}
\end{cases}
$$

    在四元数中,\emph{任意的旋转都可以由两个互为相反数的四元数表示}。取$\theta$为0,则得到一个没有任何旋转的实四元数:

$$
    q_0 = [\pm1, 0, 0, 0]^T
$$

\subsection{四元数运算}

\begin{quote}
    \centering
    四元数和通常复数一样,可以进行四则运算
\end{quote}

    现有两个四元数$q_a, q_b$,它们的向量表示为$[S_a, V_a],[S_b, V_b]$或者原始四元数$q_a = s_a + x_ai + y_aj + z_ak,q_b = s_b + x_bi + y_bj + z_bk$

\begin{itemize}
    \item 加法减法:$q_a \pm q_b = [S_a \pm S_b, V_a \pm V_b]$
    \item 乘法:
    $$\begin{aligned}
        q_aq_b &= s_as_b - x_ax_b - y_ay_b - z_az_b \\
               &+ (s_ax_b + x_as_b + y_az_b - z_ay_b)i \\
               &+ (s_ay_b - x_az_b + y_as_b + z_ax_b)j \\
               &+ (s_az_b + x_ay_b - y_ax_b + z_as_b)k
    \end{aligned}$$
    \item 使用向量形式并采用内外积运算:$q_aq_b = [s_as_b - v_a^Tv_b, s_av_b + s_bv_a + v_a \times v_b]$
    \item 共轭:$q_a^* = s_a - x_ai - y_aj - z_ak = [s_a, -v_a]$
    \item 模长:$\rVert{q_a}\rVert = \sqrt{s_a^2 + x_a^2 + y_a^2 + z_a^2}$
    \subitem 两个四元数乘积的模即为模的乘积:$\rVert{q_aq_b}\rVert = \rVert{q_a}\rVert \rVert{q_b}\rVert$ 
    \item 逆:$q^{-1} = \frac{q^*}{\rVert{q}\rVert^2}$
    \subitem 四元数与自己逆的乘积为实四元数:$qq^{-1} = q^{-1}q = 1$
    \subitem 如果q为单位四元数,其逆与共轭相等:$(q_aq_b)^{-1} = q_b^{-1}q_a^{-1}$
    \item 数乘:$q_a \cdot q_b = s_as_b + x_ax_bi + y_ay_bj + z_az_bk$
    \item 点乘:$q_a \cdot q_b = s_as_b + x_ax_bi + y_ay_bj + z_az_bk$
\end{itemize}

\subsection{用四元数表示旋转}

    我们可以用四元数表达对一个点的旋转,假设一个空间三维点$p = [x, y, z] \in R^3$,以及一个由轴角$n, \theta$指定的旋转。如果用矩阵旋转描述,那么有$p^{'} = Rp$,那么现在用四元数来表示

    首先将三维空间点用一个虚四元数描述:

$$
    p = [0, x, y, z] = [0, v]
$$

    这相当于将四元数的三个虚部与空间中的三个轴对应,然后:

$$
    q = [cos\frac{\theta}{2}, nsin\frac{\theta}{2}]
$$

    那么经过旋转后的$p^{'}$,可以表示为:

$$
    p^{'} = qpq^{-1}
$$
\end{document}

